\documentclass[12pt]{article}
\usepackage[margin=1cm]{geometry}
\usepackage{polski}
\usepackage[utf8]{inputenc}
\usepackage{siunitx}
\usepackage{amsmath}
\usepackage{graphicx}
\usepackage{multicol}
\usepackage{nopageno}
\usepackage{subcaption}

\begin{document}

\section{Niepewności pomiarowe i obliczenia}
\noindent
$\overline{\Delta h}$ - średnia grubość pozorna płytki \\
$H$ - Grubość rzeczywista
\begin{flalign}
u(H) &= \frac{0.01\si\mm}{\sqrt{3}} = 0.0058\si\mm \text{ - na podstawie działki elementarnej śruby mikrometrycznej}&&
\end{flalign}
\subsection{Szkło}
\begin{flalign}
u(H) &= 0.00043\si\mm &&\\
u(n) &= n\sqrt{\left(\frac{u(\overline{\Delta h})}{\overline{\Delta h}}\right)^2 + \left(\frac{u(H)}{H}\right)^2} = 0.0023&&
\end{flalign}
\subsection{Pleksiglas}
\begin{flalign}
u(H) &= 0.00024\si\mm &&\\
u(n) &= n\sqrt{\left(\frac{u(\overline{\Delta h})}{\overline{\Delta h}}\right)^2 + \left(\frac{u(H)}{H}\right)^2} = 0.0041&&
\end{flalign}

\section{Wnioski}
Ze względu na niedokładności metody pomiarowej (organoleptyczne ustawianie ostrości mikroskopu) wyniki odbiegają od wartości tablicowych, a błędy te są na tyle duże, że niestety uniemożliwiają pokazanie zjawiska dyspersji.
\Large\begin{flalign*}
n_{szkła} &= 1.91 \pm 0.0023\hspace{1em} n_{tab} \in (1.4, 1.9)\\
n_{pleksi} &= 1.52 \pm 0.0041\hspace{1em} n_{tab} = 1.49
\end{flalign*}

\end{document}
