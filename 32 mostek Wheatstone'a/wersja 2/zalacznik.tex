\documentclass[12pt]{article}
\usepackage[margin=1cm]{geometry}
\usepackage{polski}
\usepackage[utf8]{inputenc}
\usepackage{siunitx}
\usepackage{amsmath}
\usepackage{graphicx}
\usepackage{multicol}
\usepackage{nopageno}

\begin{document}
\section{Porównanie wartości obliczonej z wartością zmierzoną}
\subsection{Wartości zmierzone}
\begin{align*}
R_1 &= 9.05 \pm 0.18 \si{\ohm}\\
R_2 &= 16.59 \pm 0.38 \si{\ohm}\\
R_1 + R_2 &= 24.37 \pm 0.77\si{\ohm}\\
1/\left(\frac{1}{R_1} + \frac{1}{R_2}\right) &= 5.590 \pm 0.090\si{\ohm}\\
\end{align*}
\subsection{Wartości obliczone}
\begin{align*}
u(R_1 + R_2) &= \sqrt{u(R_1)^2 + u(R_2)^2} = 0.42\si{\ohm}\\
u(R_p) &= \sqrt{\left(\frac{R_2^2}{(R_1 + R_2)^2}\cdot u(R_1)\right)^2 + \left(\frac{R_1^2}{(R_1 + R_2)^2}\cdot u(R_2)\right)^2} = 0.089\si{\ohm}\\\\
R_1 + R_2 &= 25.64 \pm 0.42\si{\ohm}\\
R_p = 1/\left(\frac{1}{R_1} + \frac{1}{R_2}\right) &= 5.860 \pm 0.089\si{\ohm}\\
\end{align*}
\subsection{Wnioski}
Dla $k = 2$ wartości zmierzone i obliczone mieszczą się w granicach wyznaczanych przez niepewności pomiarowe.
\end{document}